Session identification is a common strategy used to develop metrics for web analytics and perform behavioral analyses of user-facing systems. Past work has argued that session identification strategies based on an inactivity threshold is inherently arbitrary or advocated that thresholds be set at about 30 minutes. In this work, we demonstrate a strong regularity in the temporal rhythms of user initiated events across several different domains of online activity (incl. video gaming, search, page views and volunteer contributions). We describe a methodology for identifying clusters of user activity and argue that the regularity with which these activity clusters appear implies a good rule-of-thumb inactivity threshold of about 1 hour.  We conclude with implications that these temporal rhythms may have for system design based on our observations and theories of goal-directed human activity.
