The concept of an activity session is an intuitive one, but it's surprisingly difficult to tie down a single definition of what a session \emph{is}.  A ``session'' may refer to ``(1) a set of queries to satisfy a single information need (2) a series of successive queries, and (3) a short period of contiguous time spent querying and examining results.''~\cite{jones2008beyond}

(1) is referred to, particularly in search-related literature~\cite{eickhoff2014lessons,jones2008beyond}, not as a session but as a task--a particular information need the user is trying to fulfil.  Multiple tasks may happen in a contiguous browsing period, or a single task may be spread out over multiple periods.
(2) is unclear. It may refer to a series of contiguous but unrelated queries (in which case it is identical to the third definition), or a series of contiguous queries based on the previous query in the sequence (in which case it is best understood as a sequence of tasks).
(3) is the most commonly-used definition in the literature we have reviewed~\cite{govseva2006empirical,nadjarbashi2004improving,spiliopoulou2003framework,white2010assessing}. This contrasts with the notion of \emph{task} and is the definition of ``session'' that we have chosen for this paper. It's also the definition used by the W3C~\cite{W3C1999}.

We found inspiration in thinking about how to model user session behavior in two distinct, but related threads: the empirical modeling work of cognitive science and the theoretical frameworks of human consciousness as applied to ``work activities''.

The lack of purely random distribution in the time between logged human actions has been the topic of recent studies focusing on the cognitive capacity of humans as information processing units.  Notably, Barbasi showed that, by modeling communication activities with decision-based priority queues, he could show evidence for a mechanism to explain the heavy tail in time between activities~\cite{barabasi2005origin} -- a pattern he describes as bursts of rapid activity followed by long periods of inactivity.  Wu et al. built upon this work to argue that short-message communication patterns could be better described by a ``bimodal'' distribution characterized by Poisson-based initiation of tasks and a powerlaw of time inbetween task events\cite{wu2010evidence}.

In contrast, Nardi calls out this cognitive science work for neglecting context in work patterns, motivation and community membership -- thereby inappropriately reducing a human to a processing unit in a vacuum~\cite{nardi1996context} (p21).  Instead, Nardi draws from the framework of Activity Theory (AT) to advocate for an approach to understanding human-computer interaction as a conscious procession of \emph{activities}.  AT describes an activity as a goal-directed or purposeful interaction of a subject with an object through the use of tools. AT further formalizes an \emph{activity} as a collection of \emph{actions}\footnote{We see Jones' conceptualization of tasks~\cite{jones2008beyond} as analogous to AT's conceptualization of \emph{action}.} directed towards completing the activity's goal.  Similarly, \emph{actions} are composed of \emph{operations}, a fundamental, indivisible, and unconscious movement that humans make in the service of performing an \emph{action}.

For an example application of AT, let us examine Wikipedia editing.  Our ethnographic work with Wikipedia editors suggests that it is common to set aside time on a regular basis to spend doing ``wiki-work''.  AT would conceptualize this wiki-work overall as an \emph{activity} and each unit of time spent engaging in the wiki-work as an ``activity phase'' -- though we prefer the term ``activity session''.

The \emph{actions} within an activity session would manifest as individual edits to wiki pages representing contributions to encyclopedia articles, posts in discussions and messages sent to other Wikipedia editors.  These edits involve a varied set of \emph{operations}: typing of characters, copy-pasting the details of reference materials, scrolling through a document, reading an argument and eventually, clicking the ``Save'' button.

In this work we draw from both the concepts of the \emph{operation-action-activity} heirarchy of Activity Theory and the empirical modeling strategies of cognitive science as applied to time between events.
