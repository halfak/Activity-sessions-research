User sessions have been used as behavioral measures of human-computer interaction for almost two decades, and for this reason, strategies for session identification from log data have been extensively studied~\cite{eickhoff2014lessons}.

Cooley et al.~\cite{cooley1999data} and Spiliopoulou et al.~\cite{spiliopoulou2003framework} contast two primary strategies for identifying sessions from activity logs: ``navigation-oriented heuristics'' and ``time-oriented heuristics''.

Time-oriented heuristics refer to the assignment of an inactivity threshold between logged activites to serve as a session delimiter.  The assumption implied is that if there is a break between a user's actions that is sufficiently long, it's likely that the user is no longer \emph{active}, the session is assumed to have ended, and a new session is created when the next action is performed. This is the most commonly-used approach to identify sessions, with 30 minutes serving as the most commonly used threshold~\cite{eickhoff2014lessons,spiliopoulou2003framework,ortega2010differences}.  Both threshold and approach appear to originate in a 1995 paper by Catledge \& Pitkow~\cite{catledge1995characterizing} that used client-side tracking to examine browsing behavior. In their work, they reported that the mean time between logged events 9.3 minutes.  They choose to add 1.5 standard deviations to that mean to achieve a 25.5 minutes inactivity threshold.  Over time this threshold has simplified to 30 minutes.

The utility and universality of this 30-minute inactivity threshold is widely debated; Mehrzadi \& Feitelson~\cite{mehrzadi2012onextracting} found that 30 minutes produced artefacts around long sessions, and could find no clear evidence of a global session inactivity threshold\footnote{Note that this conclusion was reached using the same AOL search dataset that we analyze in this paper}, while Jones \& Klinkner~\cite{jones2008beyond} found the 25.5 minute threshold performed ``no better than random'' in the context of intentifying search tasks. Other thresholds have been proposed, but Montgomery and Faloutsos~\cite{montgomery2001identifying} concluded that the actual threshold chosen made little difference to how accurately sessions were identified.

Navigation-oriented heuristics involve inferring browsing patterns based on the HTTP referers and URLs associated with each request by a user. When a user begins navigating (without a referer), they have started a session; when a trail can no longer be traced to a previous request based on the referers and URLs of subsequent requests, the session has ended.  This approach was pioneered by Cooley et al in 2002~\cite{cooley1999data}.  While it demonstrated utility in identifying ``tasks'', and has been extended by Nadjarbashi-Noghani et al.~\cite{nadjarbashi2004improving}, it shows poor performance on sites with framesets due to implicit assumptions about web architecture~\cite{berendt2003impact}. Further, the sheer complexity of this strategy and it's developmental focus on \emph{task} over \emph{session} make it unsuitable as a replacement for time-oriented heuristics in practical web analytics of user sessions.

In this work, we will challenge the assertion by prior works that (1) no reasonable cutoff is identifiable from the empirical data and (2) a global inactivity threshold is inappropriate as a session identification strategy.  To our knowledge, we are the first to apply a general session identification methodology to a large collection of datasets and to conclude that not only are global inactivity thresholds an appropriate strategy for session identification, but also that, for most user-initiated actions, an inactivity threshold of 1 hour is most appropriate.
